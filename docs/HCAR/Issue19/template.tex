\documentclass{scrreprt}
\usepackage{paralist}
\usepackage{graphicx}
\usepackage[final]{hcar}

\begin{document}

\begin{hcarentry}{Haskell-MPI}
\report{Bernie Pope}
\status{First public version to be released soon.}
\participants{Dmitry Astapov, Duncan Coutts}% optional
\makeheader

MPI, the \emph{Message Passing Interface}, is a popular communications protocol
for distributed parallel computing (\url{https://www.mpi-forum.org/}). It is widely
used in high performance scientific computing, and is designed to scale up from
small multi-core personal computers to massively parallel supercomputers.
MPI applications
consist of independent computing processes which share information by message passing
communication. It supports both point-to-point and collective communication operators,
and manages much of the mundane aspects of message delivery. There are several
high-quality implementations of MPI available which adhere to the standard API
specification (the latest version of which is 2.2). The MPI specification defines
interfaces for C, C++ and Fortran, and bindings are available for many other
programming languages. As the name suggests, Haskell-MPI provides a Haskell interface
to MPI, and thus facilitates distributed parallel programming in Haskell. It is implemented
on top of the C API via Haskell's foreign function interface. Haskell-MPI provides
three different ways to access MPI's functionality:
\begin{enumerate}
   \item A direct binding to the C interface.
   \item A convenient interface for sending arbitrary serializable Haskell data values as messages.
   \item A high-performance interface for working with (possibly mutable) arrays of storable
         Haskell data types.
\end{enumerate}
We do not currently provide exhaustive coverage of all the functions and types defined by MPI
2.2, although we do provide bindings to the most commonly used parts. In future we plan
to extend coverage based on the needs of projects which use the library.

We are in the final stages of preparing the first release of Haskell-MPI. We will
publish the code on Hackage once the user documentation is complete.
We have run various simple latency and bandwidth tests using up to 512 Intel x86-64 cores, and
for the high-performance interface, the results are within acceptable bounds of those
achieved by C.
Haskell-MPI is designed to work with any compliant implementation of MPI, and we
have successfully tested it with both OpenMPI (\url{https://www.open-mpi.org/}) and
MPICH2 (\url{https://www.mpich.org/}).

The source code repository can be browsed or downloaded from Github:
  \url{https://github.com/bjpop/haskell-mpi}.

\end{hcarentry}

\end{document}
